
\let\negmedspace\undefined
\let\negthickspace\undefined
\documentclass[journal,12pt,twocolumn]{IEEEtran}
\usepackage{cite}
\usepackage{amsmath,amssymb,amsfonts,amsthm}
\usepackage{algorithmic}
\usepackage{graphicx}
\usepackage{textcomp}
\usepackage{xcolor}
\usepackage{txfonts}
\usepackage{listings}
\usepackage{enumitem}
\usepackage{mathtools}
\usepackage{gensymb}
\usepackage{comment}
\usepackage[breaklinks=true]{hyperref}
\usepackage{tkz-euclide} 
\usepackage{listings}
\usepackage{gvv}                                        
\def\inputGnumericTable{}                                 
\usepackage[latin1]{inputenc}                                
\usepackage{color}                                            
\usepackage{array}                                            
\usepackage{longtable}                                       
\usepackage{calc}                                             
\usepackage{multirow}                                         
\usepackage{hhline}                                           
\usepackage{ifthen}                                           
\usepackage{lscape}

\newtheorem{theorem}{Theorem}[section]
\newtheorem{problem}{Problem}
\newtheorem{proposition}{Proposition}[section]
\newtheorem{lemma}{Lemma}[section]
\newtheorem{corollary}[theorem]{Corollary}
\newtheorem{example}{Example}[section]
\newtheorem{definition}[problem]{Definition}
\newcommand{\BEQA}{\begin{eqnarray}}
\newcommand{\EEQA}{\end{eqnarray}}
\newcommand{\define}{\stackrel{\triangle}{=}}
\theoremstyle{remark}
\newtheorem{rem}{Remark}
\begin{document}

\bibliographystyle{IEEEtran}
\vspace{3cm}

\title{Probability Assignment}
\author{EE22BTECH11042-Rajeev Kumar
}
\maketitle
\newpage
\bigskip
\renewcommand{\thefigure}{\theenumi}
\renewcommand{\thetable}{\theenumi}

Find the probability of getting 5 twice in 7 throws of a dice.
\solution
Let $X$ be random variable defined as
\begin{table}[!ht]
	\input{./tables/table.tex}
\end{table}\\
$X$ has a binomial distribution with parameters
\begin{align}
n=7 \qquad p=\frac{1}{6}
\end{align}
Pmf of $X$ for $1 \leq k \leq 7$ is
\begin{align}
p_X(k)&=\comb{n}{k}p^k(1-p)^{n-k}
\end{align}
Probability of getting 5 twice in 7 throws of a dice is given by:
\begin{align}
p_X(2)&=\comb{7}{2}\left(\frac{1}{6}\right)^2\left(1-\frac{1}{6}\right)^{7-2}\\
&=\frac{7!}{5!2!}\left(\frac{1}{6}\right)^2\left(\frac{5}{6}\right)^{5}\\
&=21\left(\frac{1}{36}\right)\left(\frac{3125}{7776}\right)\\
&=21\left(\frac{3125}{279936}\right)\\
&=0.234\label{eq:9.3.13.1}
\end{align}
Let Y be goussian variable
\begin{align}
\mu&=np\\
&=\frac{7}{6}
\end{align}
\begin{align}
\sigma^2&=np(1-p)\\
&=\frac{35}{36}
\end{align}
Using Normal distribution at X=2.
\begin{align}
Z&=\frac{X-\mu}{\sigma}\\
&=\frac{2-\frac{7}{6}}{\sqrt{\frac{35}{36}}}\\
&=0.845
\end{align}
For pdf calculation
\begin{align}
f_Y(x)&=\frac{1}{\sqrt{2\pi\sigma^2}}e^{-\frac{(x-\mu)^2}{2\sigma^2}}
\end{align}
\begin{figure}[!ht]
\centering
\includegraphics[width=\columnwidth]{./figs/figure.png}
\caption{Binomial pmf vs Gaussian pdf }
\label{fig:9.3.13}
\end{figure}
From the plot, pmf is close to normal distribution pdf.
\begin{align}
	p_Y(2)&=p_Z(0.845)\\
&=0.234\label{eq:9.3.13.2}
\end{align}
From \eqref{eq:9.3.13.1} and \eqref{eq:9.3.13.2},
\begin{align}
p_X(2)\approx p_Y(2)
\end{align}
\end{document}
